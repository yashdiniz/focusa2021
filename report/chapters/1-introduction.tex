\chapter{Introduction}

% restarting page numbering from Introduction as stated in project report guidelines.
\pagenumbering{arabic}

The essence of education is the ability for teachers to efficiently 
disseminate knowledge and obtain positive feedback from students, by 
using tools for a good teaching-learning experience. 
Educational environments have a lot of existing tools which help students 
and teachers interact and share resources. These tools, as will be explained 
further in this report, are developed to serve very specific and limited 
purposes, and are difficult to manage manually. They fail to provide a 
single comprehensive platform for all necessary operations. 
Moreover, these existing tools fail to cater to the management and analysis 
of data and thus are not used for generating insights and management optimisation. 
Furthermore, our worldwide internet infrastructure is lacking the reliability it 
needs to help people stay connected~\cite{WebF20}.

\section{Problem Definition}

Multiple e-learning platforms exist, each trying to solve a unique problem. 
Very few platforms function using an offline-first architecture, thus leading to 
really high dependency on network infrastructure. 
It gets difficult for both students and teachers to manage their work, 
since they have to use multiple platforms, dividing their attention among each of them. 
Current e-learning platforms also provide very little in the way of automation, 
and lack a common, integrable interface.

Not all software platforms are built considering unreliable internet service. 
This dependency on a reliable internet by a majority of its users thus affects the entire network, 
and reduces quality of service for all internet users.This problem can be solved in many ways, 
like improving the network infrastructure, or building applications that 
responsibly use internet bandwidth. If unattended, issues of an unreliable 
system can include lesser productivity.

\subsection{Existing Systems}

There are a lot of e-learning systems and platforms available in the market 
but the most popular ones are Google Classroom and Moodle. 
The following shall be a brief overview of these popular e-learning platforms.

\subsubsection{Google Classroom}
% Pranav Paranjape
To quite an extent google classroom managed to replace the traditional 
learning management system with no paper requirement, easy document sharing using google docs, 
easy assignment submission process and features to integrate with other Google services 
like YouTube, Drive and even Meet. 
Google Classroom couldn't fully become a replacement for education systems because of the following:

\begin{itemize}
    \item \textbf{Difficult account management.} Suppose the user wants to upload an assignment onto the classroom for submission, 
    which is currently on another google account, the user will need to logout of the current account, 
    download the document and sign in with the required Google account and upload; quite a hassle.
    \item \textbf{Doesn’t have an offline first design}, which eliminates the need of always being online to access the documents, 
    as the documents can be downloaded in the user’s local storage for a limited amount of time.
    \item \textbf{No live post updates}, i.e. the user need to keep refreshing to view the recent updates.
\end{itemize}

One biggest pros of google education services is modularity. The G-Suite offers all these 
loosely coupled services like Meet, Classroom, Hangouts, and many others.

\subsubsection{Moodle}
% Pranav Paranjape
Yet another popular e-learning platform and the best alternative to Google Classroom. 
Its pros include limited offline use for certain features. It has impressive features 
for  uploading and downloading lecture notes, creating quizzes and tests, supports push 
notifications for both students and teachers, generating reports and many others. 
It has a  backup, restore and import features, which turn out to be really useful for teachers. 
The teachers can also manage learners' profiles and setting enrollment keys, with role-based restrictions.

One of it’s biggest cons however include poor scalability, due to its tightly coupled nature. 
This also causes issues with robustness, since a single module crash can cause the whole system to crash. 
No inbuilt video-conferencing functionalities, and it also becomes difficult to integrate a 
third party service due to the same tightly coupled nature.

It also offers a relatively poor mobile offering, with limited integration into third-party modules. 
There is a major learning curve for building and taking Moodle administration courses for beginners.

\subsubsection{Active Document Platform}
% Yash Diniz
The Active Document Platform(AD)~\cite{ActiveDocument} is a very useful platform that 
solves the problem of unreliable internet connectivity, by serving as an offline-first 
and distributed way of sharing course documents. Not being dependent on a centralised 
server on the internet allows it to operate offline as well, and it synchronizes with 
its main node whenever internet connectivity gets established. The best part about AD 
is its document organization, and distributed collaborative editing. Furthermore, AD 
uses SCORM (Sharable Content Object Reference Model), which is a collection of standards 
and APIs which help streamline connecting between other e-learning platforms which also 
use SCORM. One of its biggest cons however is limited functionality, and not very user-friendly.

\subsection{Proposed System}

Our project aims to deliver a scalable web framework which is easy to work with 
for both the developer and the customers. It will allow connecting students and teachers 
through subscription portals, by letting faculty moderators post content which can be accessed 
by the students and synced whenever an internet connection is available. 
Furthermore, the project will have a real-time video conferencing platform 
using an experimental lossy compression algorithm involving neural networks. 
Finally, various existing services like Google Calendar and Drive will be integrated, to ease 
automation efforts, reducing the need to manually manage multiple tools.

\section{Purpose of the Project}

The purpose of the project is to offer a single integrated interface which the students can access and stay updated.
Furthermore, the project will adopt an offline-first architecture, thus reducing the dependency on network infrastructure.
Being an offline-first application would help the students access their work while being "disconnected", which 
can help the students focus when they need to.

The project will also prioritize optimizing for scalability, robustness and flexibility, allowing easy integration 
for new modules. One of the goals of this project is to improve communication between users of the product while reducing 
network bandwidth usage.
Finally, the project aims to build interfaces and tools for anonymous data collection and analysis for measuring,
managing, and possibly automating various patterns, while also offering insights to the faculty.

\section{Scope of the Project}
% Alston Dias
\subsubsection{Must be implemented}
\begin{itemize}
    \item Scalable and loosely coupled Web Services Framework.
    \item Connecting students and teachers through subscription portals (courses), 
    which can be made private by virtue of an authentication code if needed.
    \item Allowing faculty and students to have specific roles, which are moderated by limited people.
\end{itemize}

\subsubsection{Should be implemented}
\begin{itemize}
    \item Integrating various existing services, like Google Calendar, Drive, classroom, etc. 
    by building web-hooks, which allow for easy automation. (by virtue of bots)
    \item A real-time video conferencing platform using an experimental lossy compression algorithm involving neural networks.
\end{itemize}

\subsubsection{Could be implemented}
\begin{itemize}
    \item Having a marketplace for modules, where it is as easy as integrating by pressing install.
    \item Creating a simple student planner application that can help the student manage submissions, 
    projects, and time in general (integrated with Calendar for ease of use)
    \item A system that can perform collection, monitoring, reporting, and long-term benefit analysis 
    of student and employee attendance at the college. It can help students set their priorities.
    \item An online test tool that allows faculty to create tests and students can answer tests.
\end{itemize}

\section{Report Organization}

The current introductory section provides a brief introduction to each chapter.

\textbf{Chapter 1: Introduction}

This section focuses on the purpose and scope of the proposed system of FOCUSA.

\textbf{Chapter 2: Literature Survey}

This section describes the concepts and technologies used to develop the project.

\textbf{Chapter 3: Software Requirement Specification}

This section provides information about the specific requirement of the proposed system.

\textbf{Chapter 4: Design}

This section describes the software lifecycle model, which will be used in developing the
software. It also includes system design and detailed design.

\textbf{Chapter 5: Implementation}

This section deals with the implementation of the project where in the snapshots of each execution steps are shown.

\textbf{Chapter 6: Conclusion}

This section deals with the conclusion that can be derived after implementing the final System.
