\chapter{Software Requirement Specification}

\section{Introduction}

\subsection{Background}

Colleges, schools and other educational environments have a lot of existing tools 
which help students and teachers interact and share resources.
These tools however are developed to serve very specific and limited purposes, and are very difficult to 
manage manually. They fail to provide a single comprehensive platform for all 
necessary management operations. Moreover, these existing tools fail to cater to the 
management and analysis of data and thus are not used for generating useful 
insights.

\subsection{Project Overview}

This project aims to deliver a scalable web services framework which is easy to work 
with for both the developer and the customers. It will allow connecting students and 
teachers through subscription portals, by letting faculty moderators post content 
which can be accessed by the students and synced whenever an internet connection 
is available. Furthermore, the project will have a real-time video conferencing platform 
using an experimental lossy compression algorithm involving neural networks. 
Finally, various existing services like Google Calendar and Drive will be integrated, to ease 
automation efforts, reducing the need to manually manage multiple tools.

\subsection{Hardware Requirements}

\textbf{Development:} 
Minimum 8 GB RAM for native app development.

\textbf{Production:} 
As stated in the constraints the server should have at least 1 GB RAM.
The client should be able to run the minimal version of the neural network encoder.
(final production requirements will be obtained through experimentation)

\subsection{Software Requirements}

\textbf{Recommended:} 
Linux OS 4.19 kernel or later in production.
Windows 10 1803 (Build 17134 or later) for development.
NGINX server 1.17 or later as reverse proxy and load balancer.
Any smartphone or PC supporting smooth operation of the most recent 
Google Chrome or Mozilla Firefox available at client.

\subsection{Constraints}

Needs to run in ~1GB, with 1 vCPU core at production environment.
Clients will have limited network bandwidth, mostly using mobile web browsers, but 
have persistent cache for offline storage (which may be cleared often)

\subsection{Assumptions}

\begin{itemize}
    \item Clients will mostly operate offline, with syncing times usually after working hours.
    \item Server will be operating in a scalable and distributed architecture, 
    initially using college or cloud infrastructure.
    \item Clients will operate offline-first using persistent browser cache to download and store 
    filtered versions of the database on themselves, effectively reducing server load.
\end{itemize}

\subsection{Dependencies}

NodeJS, GraphQL, CouchDB, ReactJS

\section{Functional Requirements}

\begin{itemize}
    \item Connecting students and teachers through subscription portals (topics), 
    which can be made private by virtue of an authentication code if needed.
    \item Allowing faculty and students to post content to specific groups, 
    which are moderated by limited people.
    \item A real-time video conferencing platform using an Experimental 
    lossy compression algorithm involving neural networks
\end{itemize}

\section{Non Functional Requirements}

\subsection{Scalability}
\begin{itemize}
    \item Using CouchDB as a distributed database allows for simplifying 
    horizontal scalability using eventual consistency.
    \item Using containerized microservices further improve the scope for horizontal scalability.
    \item Simple since GraphQL is used as the Web Services Framework 
    and also using the ReactJS and CouchDB technology stack.
\end{itemize}

\subsection{Portability}
\begin{itemize}
    \item GraphQL is a platform-independent querying API, 
    which can serve as a Web Services Framework for improved portability and integration.
    \item ReactJS is a cross-platform framework for building reactive websites. 
    It supports any platform that can run a modern graphical web-browser.
\end{itemize}

\subsection{Security}
\begin{itemize}
    \item SSL will be used for encrypting the traffic.
    \item Servers will have firewalls, and SSH connections will be protected by password and certificates.
    \item Each service will also be containerized, running in complete isolation, interacting strictly via 
    loosely coupled message passing.
\end{itemize}

\subsection{Maintainability}
\begin{itemize}
    \item Microservice architecture will be used on the server-side. 
    The loosely-coupled modularity offered by this architecture simplifies maintenance tasks.
    \item The frontend will be built with reusable ReactJS components. 
    The goal is to maximize code reuse, and simplify maintenance.
    \item Third-party npm packages will also be incorporated wherever necessary.
\end{itemize}

\subsection{Performance}
\begin{itemize}
    \item Distillation techniques for the neural networks will also be incorporated,
    and an attampt will be made to reduce the CPU and memory footprints of the same as much as possible.
\end{itemize}

\section{Interface Requirements}

\subsection{User Interfaces}
\begin{itemize}
    \item Simple, minimal, responsive and easy to navigate User Interface.
    \item User Interface elements must be consistent.
    \item Strategic use of themes and colors to suit the purpose.
    \item Component to navigate to and display the content pertaining to video conferencing module.
    \item Components to navigate to course subscription, profile and post, login, logout.
    \item Components dealing with posting, collection and visual display of data analysis
\end{itemize}

\subsection{Hardware Interfaces}
\begin{itemize}
    \item SSH protocol for the server-side interface.
    \item Device hardware, like the camera, microphone, and others. 
    The WebRTC negotiation interface will be used to obtain secure access to the hardware.
\end{itemize}

\subsection{Communication Interfaces}
\begin{itemize}
    \item WebRTC (Real-Time Communication) standards and protocols to enable real time, peer to peer audio and video communication.
    \item Request/Response protocol, HTTP/1.1 (as inherent to GraphQL)
\end{itemize}

\section{Technology Used}

\textbf{Frontend}: ReactJS (For Webapp), React Native (for Mobile App)

\textbf{Middleware}: NodeJS, GraphQL

\textbf{Backend}: CouchDB, PouchDB databases, with promises.

\section{Definitions, Acronyms and Abbreviations}

\begin{itemize}
    \item \textbf{Web Service} - a service offered by an electronic device to another electronic device, 
    communicating with each other via the World Wide Web, or a server running on a computer device.
    \item \textbf{Native App} - An app to be installed and run on a device without using any form of emulation.
    \item \textbf{Neural Network Encoder} - A neural network encoder converts input into a feature vector 
    which represents the input but taking lesser space than the original input.
    \item \textbf{Reverse Proxy} - ensures smooth flow of traffic between the client and the server 
    and routes the client requests to appropriate backend server. Popularly used for load balancing and security.
    \item \textbf{Load Balancing} - methodical and efficient distribution of network or application traffic 
    across multiple servers.
    \item \textbf{Microservices} - The application is broken down into multiple, isolated services, each performing 
    a small part of the entire application. Microservices are easy to manage and operate in parallel.
    \item \textbf{Message Passing} - Applications can share data with each other by passing messages over the network. 
    These messages will be passed using a standard protocol like HTTP.
    \item \textbf{Persistent Cache} - intended for intermediate term storage of documents or data objects.
    \item \textbf{Lossy Compression} - method of data compression in which the size of the file 
    is reduced by eliminating data in the file.
    \item \textbf{SSH} - Secure Shell Protocol is a method for secure remote login from 
    one computer to another.
    \item \textbf{WebRTC} - Web Real-Time Communication comprises of protocols, standards and 
    JavaScript API which enable real time P2P communication.
    \item \textbf{Peer to Peer} - Application to Application communication over the internet, 
    where data packets always attempt to find the shortest path between devices, thus possibly reducing latency.
    \item \textbf{HTTP} - Hypertext Transfer Protocol gives users a way to interact with web resources 
    such as HTML files by transmitting hypertext messages between clients and servers.
\end{itemize}
