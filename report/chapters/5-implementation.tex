\chapter{Implementation}

% Just added a stub.
% TODO: Complete and fill up the implementation chapter!

% Include experiments and observations from the neural network experiment.
% Also include the process of creating microservices: 1. CRUD. 2. REST. 3. GraphQL.
% Also include the methods of test cases performed.
% Also include the Kubernetes and other deployment stuff.
% Talk about how we are implementing the couchDB distributed DBs.


\section{Overview of the Technologies used}

\subsection{Backend Technologies}

    \subsubsection{PouchDB}
    PouchDB is an open-source JavaScript database inspired by Apache CouchDB 
    that is designed to run well within the browser.
    It enables applications to store data locally while offline, 
    then synchronize it with CouchDB and compatible servers when the application is back online, keeping the user's data in sync 
    no matter where they next login.

    \subsubsection{RxDB}
    RxDB (short for Reactive Database) is a NoSQL-database for JavaScript Applications like Websites, 
    hybrid Apps, Electron-Apps, Progressive Web Apps and NodeJs. Reactive means that you can not only 
    query the current state, but subscribe to all state changes like the result of a query or even a 
    single field of a document. This is great for UI-based realtime applications in way that makes it 
    easy to develop and also has great performance benefits. To replicate data between your clients and server,
    RxDB provides modules for realtime replication with any CouchDB compliant endpoint and also with custom GraphQL endpoints. 

    \subsubsection{NodeJS}
    Node. js is a platform built on Chrome's JavaScript runtime for easily building fast and scalable network applications. 
    Node. js uses an event-driven, non-blocking I/O model that makes it lightweight and efficient, perfect for data-intensive 
    real-time applications that run across distributed devices.

    \subsubsection{Express JS}
    Express.js, or simply Express, is a back end web application framework for Node.js, released as free and open-source software under the MIT License. 
    It is designed for building web applications and APIs.
    It has been called the de facto standard server framework for Node.js.

    \subsubsection{GraphQL}
    GraphQL is a query language for APIs and a runtime for fulfilling those queries with your existing data. 
    GraphQL provides a complete and understandable description of the data in your API, gives clients the power to ask for exactly what they need and nothing more, 
    makes it easier to evolve APIs over time, and enables powerful developer tools.

    \subsubsection{Apollo Server}
    Apollo Server is an open-source, spec-compliant GraphQL server that's compatible with any GraphQL client, including Apollo Client. 
    It's the best way to build a production-ready, self-documenting GraphQL API that can use data from any source.

    \subsubsection{Apollo Client}
    Apollo Client is a comprehensive state management library for JavaScript that enables you to manage both local and remote data with GraphQL.
    Use it to fetch, cache, and modify application data, all while automatically updating your UI.
    Apollo Client helps you structure code in an economical, predictable, and declarative way that's consistent with modern development practices. 

    \subsubsection{Apollo Link}
    The Apollo Link library helps you customize the flow of data between Apollo Client and your GraphQL server.

    \subsubsection{Docker}
    Docker is an open platform for developing, shipping, and running applications. 
    Docker enables you to separate your applications from your infrastructure so you can deliver software quickly. 
    With Docker, you can manage your infrastructure in the same ways you manage your applications. 
    By taking advantage of Docker’s methodologies for shipping, testing, and deploying code quickly, 
    you can significantly reduce the delay between writing code and running it in production.

    \subsubsection{IPFS}
    The InterPlanetary File System (IPFS) is a protocol and peer-to-peer network for storing and sharing data in a distributed file system. 
    IPFS uses content-addressing to uniquely identify each file in a global namespace connecting all computing devices.

    \subsubsection{Multer}
    Multer is a node. js middleware for handling multipart/form-data , 
    which is primarily used for uploading files. It is written on top of busboy for maximum efficiency.

\subsection{Front-end Technologies}

    \subsubsection{Redux}
    Redux is a predictable state container for JavaScript apps.
    It helps you write applications that behave consistently, run in different environments 
    (client, server, and native), and are easy to test. On top of that, it provides a great developer experience, 
    such as live code editing combined with a time traveling debugger.
    
    \subsubsection{React JS}
    React is a declarative, efficient, and flexible JavaScript library for building user interfaces. 
    It lets you compose complex UIs from small and isolated pieces of code called “components”.

    \subsubsection{React Native}
    React Native is an open-source mobile application framework created by Facebook, Inc.
    It is used to develop applications for Android, Android TV, iOS, macOS, tvOS, Web, Windows and UWP by enabling 
    developers to use React's framework along with native platform capabilities.

    \subsubsection{Expo}
    Expo is a framework and a platform for universal React applications. 
    It is a set of tools and services built around React Native and native platforms that help you develop, build, 
    deploy, and quickly iterate on iOS, Android, and web apps from the same JavaScript/TypeScript codebase.

\subsection{Other Technologies}

    \subsubsection{WebRTC}
    With WebRTC, you can add real-time communication capabilities to your application that works on top of an open standard. 
    It supports video, voice, and generic data to be sent between peers, allowing developers to build powerful 
    voice- and video-communication solutions. The technology is available on all modern browsers as well as on native clients 
    for all major platforms. The technologies behind WebRTC are implemented as an open web standard and available as regular JavaScript APIs 
    in all major browsers. For native clients, like Android and iOS applications, a library is available that provides the same functionality.

    \subsubsection{Jitsi}

    Jitsi is a collection of free and open-source multiplatform voice (VoIP), video conferencing and instant messaging applications for the web platform, 
    Windows, Linux, macOS, iOS and Android.

    \subsubsection{TensorFlow}
    TensorFlow is an end-to-end open source platform for machine learning. It has a comprehensive, flexible ecosystem of tools, 
    libraries and community resources that lets researchers push the state-of-the-art in ML and developers easily build and deploy ML powered applications.

    \subsubsection{Keras}
    Keras is an open-source software library that provides a Python interface for artificial neural networks. Keras acts as an interface for the TensorFlow library. 