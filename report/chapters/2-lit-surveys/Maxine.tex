\subsection{NVIDIA Maxine}

NVIDIA Maxine is a fully accelerated platform SDK for developers of video 
conferencing services to build and deploy AI-powered features that use state-of-the-art 
models in their cloud. Video conferencing applications based on Maxine can reduce video bandwidth usage down to one-tenth of H.264 using AI video compression, dramatically reducing costs.

Maxine includes APIs for the latest innovations from NVIDIA research such as face alignment, 
gaze correction, face re-lighting, and real-time translation in addition to capabilities such as super-resolution, noise removal, closed captioning, and virtual assistants. These capabilities are fully accelerated on NVIDIA GPUs to run in real-time video streaming applications in the cloud.

Maxine-based applications let service providers offer the same features to every user on any device,
including computers, tablets, and phones. Applications built with Maxine can easily be deployed as 
microservices that scale to hundreds of thousands of streams in a Kubernetes environment.~\cite{Maxine}

\subsubsection{Features}

\begin{itemize}
    \item \textbf{Easy to use SDK}: Includes libraries, tools and example pipelines 
    for developers to quickly add AI features to their applications.
    \item \textbf{Ultra-low Bandwidth}: AI Video Compression uses one-tenth the 
    bandwidth of H.264 video compression standard.
    \item \textbf{State-of-the-art AI model}: Includes pre-trained models with thousands of hours 
    of training on NVIDIA DGX\texttrademark A100.
    \item \textbf{Fully GPU Accelerated}: Optimizes end-to-end pipelines for the highest performance 
    on NVIDIA Tensor Cores GPUs.
\end{itemize}

\subsubsection{Key Technologies}

\begin{itemize}
    \item \textbf{Reduce Video Bandwidth compared to H.264}: 
    With AI-based video compression technology running on \textbf{NVIDIA GPUs}, 
    developers can reduce bandwidth use down to one-tenth of the bandwidth needed 
    for the H.264 video compression standard. This cuts costs for providers and 
    delivers a smoother video conferencing experience for end-users, who can enjoy 
    more AI-powered services while streaming less data on their computers, tablets, and phones.
    
    \item \textbf{Face Re-Animation}:
    Using new AI research, the key facial points of each person on a video call can be identified
    and then used to reanimate a person’s face on the other side using a still image
    of the call with Generative Adversarial Networks (GANs).
    
    These key points can be used for face alignment, where faces are rotated so that
    people appear to be facing each other during a call, as well as gaze correction 
    to help simulate eye contact, even if a person’s camera isn’t aligned with their screen.
    
    Developers can also add features that allow call participants to choose their
    own avatars that are realistically animated in real-time by their voice and emotional tone.

    \item \textbf{Video and Audio Effects}:
    AI-based super-resolution and artifact reduction can convert lower resolutions to higher
    resolution videos in real-time which help to lower the bandwidth requirements for video 
    conference providers, as well as improve the call experience for users with lower bandwidth. 
    Developers can add features to filter out common background noise and frame the camera on a user’s 
    face for a more personal and engaging conversation.
    
    Additional AI models can help remove noise from low-light conditions creating a more appealing picture.

    \item \textbf{Conversational AI}:
    Maxine-based applications can use NVIDIA Jarvis, a fully accelerated conversational
    AI framework with state-of-the-art models optimized for real-time performance. Using Jarvis, 
    developers can integrate virtual assistants to take notes, set action items, and answer questions in human-like voices.
    
    Additional conversational AI services such as translations, closed captioning and transcriptions help ensure everyone can understand what’s being discussed on the call.
\end{itemize}

\subsubsection{Observation}

NVIDIA Maxine needs all of its devices to have NVIDIA hardware or access to the NVIDIA cloud. 
This reduces the customer base because of its cost and the huge unavailability of GPUs.

It is therefore important to understand that dependence on compute-intensive hardware could
currently, be the only barrier to building effective applications using Neural Networks.

\subsection{Comfort Noise}

Comfort noise is an audibly soft, synthetic background noise that was introduced in wireless communications 
mainly to overcome the issues of artificial silence which occurs in voice activity detection systems.
The issues of receiving prolonged periods of artificial silence would include:

\begin{itemize}
    \item The listener disconnecting prematurely believing transmission has been lost.
    \item The speech does not sound smooth, since the voice activity detector would abruptly cut the signal as an attempt to save bandwidth, but this would give the listener a perception that the voice quality has worsened.
    \item The sudden changes in voice amplitude, punctuated by repeated gaps of silence would be jarring to the listener.
\end{itemize}

As can be observed, the above issues are not technical. Engineers arrived at a simple solution that involved adding comfort noise to the communication. This noise is added at the receiving end of the communication, and not transmitted.~\cite{ComfortNoise}
Such a setup helps save bandwidth while also circumventing the issues described above.